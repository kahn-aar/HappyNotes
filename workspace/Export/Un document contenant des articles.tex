\documentclass[a4paper,11pt]{report}
\usepackage{graphicx}
\usepackage[french]{babel}
\usepackage[latin1]{inputenc}
\usepackage[T1]{fontenc}
\begin{document}
\chapter{Un document contenant des articles}
\section{article 1}
\paragraph{Article}Post haec indumentum regale quaerebatur et ministris fucandae purpurae tortis confessisque pectoralem tuniculam sine manicis textam, Maras nomine quidam inductus est ut appellant Christiani diaconus, cuius prolatae litterae scriptae Graeco sermone ad Tyrii textrini praepositum celerari speciem perurgebant quam autem non indicabant denique etiam idem ad usque discrimen vitae vexatus nihil fateri conpulsus est.
Quam ob rem id primum videamus, si placet, quatenus amor in amicitia progredi debeat. Numne, si Coriolanus habuit amicos, ferre contra patriam arma illi cum Coriolano debuerunt? num Vecellinum amici regnum adpetentem, num Maelium debuerunt iuvare?
Tempore quo primis auspiciis in mundanum fulgorem surgeret victura dum erunt homines Roma, ut augeretur sublimibus incrementis, foedere pacis aeternae Virtus convenit atque Fortuna plerumque dissidentes, quarum si altera defuisset, ad perfectam non venerat summitatem.
\section{article 2}
\paragraph{Article}Nam quibusdam, quos audio sapientes habitos in Graecia, placuisse opinor mirabilia quaedam (sed nihil est quod illi non persequantur argutiis): partim fugiendas esse nimias amicitias, ne necesse sit unum sollicitum esse pro pluribus; satis superque esse sibi suarum cuique rerum, alienis nimis implicari molestum esse; commodissimum esse quam laxissimas habenas habere amicitiae, quas vel adducas, cum velis, vel remittas; caput enim esse ad beate vivendum securitatem, qua frui non possit animus, si tamquam parturiat unus pro pluribus.
Et quoniam mirari posse quosdam peregrinos existimo haec lecturos forsitan, si contigerit, quamobrem cum oratio ad ea monstranda deflexerit quae Romae gererentur, nihil praeter seditiones narratur et tabernas et vilitates harum similis alias, summatim causas perstringam nusquam a veritate sponte propria digressurus.
\end{document}
