\documentclass[a4paper,11pt]{report}
\usepackage{graphicx}
\usepackage[french]{babel}
\usepackage[latin1]{inputenc}
\usepackage[T1]{fontenc}
\begin{document}
\chapter{article 1}
\paragraph{article 1} Post haec indumentum regale quaerebatur et ministris fucandae purpurae tortis confessisque pectoralem tuniculam sine manicis textam, Maras nomine quidam inductus est ut appellant Christiani diaconus, cuius prolatae litterae scriptae Graeco sermone ad Tyrii textrini praepositum celerari speciem perurgebant quam autem non indicabant denique etiam idem ad usque discrimen vitae vexatus nihil fateri conpulsus est.
Quam ob rem id primum videamus, si placet, quatenus amor in amicitia progredi debeat. Numne, si Coriolanus habuit amicos, ferre contra patriam arma illi cum Coriolano debuerunt? num Vecellinum amici regnum adpetentem, num Maelium debuerunt iuvare?
Tempore quo primis auspiciis in mundanum fulgorem surgeret victura dum erunt homines Roma, ut augeretur sublimibus incrementis, foedere pacis aeternae Virtus convenit atque Fortuna plerumque dissidentes, quarum si altera defuisset, ad perfectam non venerat summitatem.
\end{document}
